\documentclass[12pt,a4paper]{report}
\linespread{1.25} 

\usepackage[utf8]{inputenc}
\usepackage[french]{babel}
\usepackage{graphicx}

\author{Benjamin PAJUSCO et Paul ALNET}
\title{Systèmes électoraux: enjeux et  emplois au sein des démocraties}
\date{2020}

% Pour citer sans que cela n'aparaisse on utilise \nocite{bibid}
% On pourra éventuellement les changer automatiquement en \cite

% For the title background, via https://tex.stackexchange.com/questions/46280/how-to-create-a-background-image-on-title-page-with-latex
\usepackage{eso-pic}
\newcommand\BackgroundPic{%
	\put(0,-180){%
		\parbox[b][\paperheight]{\paperwidth}{%
			\vfill
			\centering
			\includegraphics[width=\paperwidth,height=\paperheight,%
			keepaspectratio]{background.jpg}%
			\vfill
			
}}
	\put(100,36){\tiny{Photo par @element5digital sur Unsplash, légèrement altérée}}
}

\begin{document}
\AddToShipoutPicture*{\BackgroundPic}
\maketitle

\textbf{Quelles sont les différentes approches pour donner la voix au peuple, leurs évolutions et quelles sont les conséquences de leurs implémentations?}

\section*{Introduction}
 L'avenant des gouvernements, la transition progressive vers des états plus démocratiques, le pouvoir théoriquement donné au au peuple, nécessite l'apport d'une méthode pour permettre à chaque citoyen de faire entendre sa voix, son opinion. \nocite{wiki:demo}
 Chacun devant vaquer à ses occupations et ne pouvant pas investir temps et ressources dans la vie démocratique ainsi que la soif accrue du pouvoir détenue par certains font que l'on opte plutôt pour désigner des représentants du peuple, souvent par le biais d'élections; on appelle ceci la démocratie représentative. \nocite{wiki:seppv}
 Aristote déjà conçoit que celle-ci doit s'accompagner d'une séparation des pouvoirs. 
 C'est à dire que les pouvoirs législatif, qui établit la loi et le budget de l'État, exécutif, qui applique la loi, et judiciaire, qui veille à la bonne application de la loi, sont partagés par différents groupes d'individus, au contraire d'une autocratie. \nocite{viepub:seppv}
 
 Ces personnes n'apparaissent pas magiquement, des procédés, variant selon les pays et régions, sont dictés dans les lois et sont suivis pour déterminer qui représentera, qui dirigera la nation.
 Nombreuses sont les organisations plaçant à la tête du pouvoir exécutif une hiérarchie menée par une personne élue, souvent président ou premier ministre, nommant les autres membres de la tête de l'administration. Il est généralement admis que cette personne dirigeante devrait être élue par le peuple, démocratiquement. 
 Ce défi logistique est confronté avec diverses approches selon les pays.
 
 \nocite{viepub:scrutins}
 
   Les modèles des démocraties modernes ne sont pas apparues instantanément mais découlent de plus de deux millénaires d'évolution.

TODO peaufiner

\tableofcontents

\chapter{La naissance de la démocratie et des systèmes électoraux}

\section{Démocratie Athénienne}
La capitale de la Grèce antique Athènes est reconnue comme l'une des premières origines de démocratie. 
En effet, au VI\up{e} siècle avant notre ère, la cité est, comme tant d'autres, confrontée à une crise politique. 
Pour y pallier, les législateurs renforcent le caractère démocratique du régime et rendent plus accessible la politique aux citoyens moins aisés en proposant notamment des compensations monétaires.

\subsection{Le corps électoral}
Si l'agglomération athénienne comptait 250 000 habitants\nocite{persee:popu}, seuls les citoyens peuvent voter. 
Or les femmes et esclaves n'étaient considérés que comme mineurs, seuls les fils de citoyen ayant suivi l'équivalent du service militaire moderne peuvent jouir du titre de citoyen.

Ces restrictions laissent un corps de jusqu'à 40 000 citoyens, composé d'hommes majeurs s'étant montrés dédiés à la cité, autorisé à participer activement à sa vie politique et décider collectivement de son sort. 
Cependant, la colline de la Pnyx ne peut accueillir qu'environ 6000 grecs, ce qui est aussi le quorum nécessaire au bon fonctionnement de l'assemblée. \nocite{wiki:pnyx}

\subsection{L'Assemblée de la colline}
Ce groupe se rassemble ponctuellement, à raison d'une quarantaine de réunions par an, à l'Assemblée de l'Ecclésia. \nocite{wiki:ecclesia}
Les citoyens présents y votent les lois, décident du budget et de l'entrée en guerre, tout comme les parlementaires des assemblées législatives dans bon nombre de démocraties modernes, mais tire au sort également les têtes du gouvernement ainsi la branche judiciaire. \nocite{wiki:heliastes}
On a ici un mode assez atypique de désigner les membres du gouvernement; on espère que le hasard fait bien les choses.

Cependant même tous les citoyens ne sont pas égaux quant aux rôles qu'il peuvent occuper.
Les citoyens les plus pauvres, les thètes et les zeugites, membres des deux dernières classes censitaires et composant la majorité des citoyens, ne sont que permis d'être élevés aux fonctions les plus élémentaires\nocite{wiki:thetes} tandis que les plus riches, tels que les pentacosiomédimnes, sont éligibles aux magistratures les plus importantes.\nocite{wiki:penta}
Ces droits sont repris dans les forces militaires où seuls les plus aisés peuvent occuper des positions de commandement.\nocite{aristote:constitathenes}

De cette manière la politique athénienne est présidée par les classes plus élevées au détriment de la majorité.

\nocite{wiki:histdemo}
\subsection{Implémentation de la majorité simple à Athènes}
Il s'agit d'une méthode très simple de rendre compte de l'avis des électeurs pour répondre à une question binaire, bien que l'on puisse parfois s'abstenir.
On comptabilise le nombre de votants en faveur du premier choix, si cette valeur est supérieure à la moitié du nombre de votants, ce choix rentre en vigueur.
Les référendums de la V\up{ème} République Française suivent généralement cette démarche.

L'implémentation athénienne est cependant là aussi davantage favorable aux plus puissants.
En Effet, les citoyens les plus précaires ont déjà la barrière économique à franchir afin de se rendre à l'Ecclésia. Le temps c'est de l'argent et ils doivent pouvoir se passer d'une journée de rémunération ou de travail dans les champs tandis que les plus fortunés peuvent facilement faire sans et la politique leur revient donc plus accessible. Outre cela, les votes se font à main levée et non anonymement. Il n'était pas rare que les plus faibles, sous la pression des plus forts, votent à contre-cœur et parfois contre leurs propres intérêts. De cette manière les rênes du pouvoir demeurent partiellement limités à une minorité.

\paragraph{} % Skips line
Ainsi le système athénien semble assez primitif face au démocratie modernes, bien qu'une bonne initiative et un bon pas vers une démocratie pure, le pouvoir reste celé entre les mains des plus puissants mais laisse les autres s'impliquer davantage.


\section{Le système romain}

\section{La I\up{ère} République Française et sa portée}
La France, pays des droits de l'Homme, fut un pilier de la popularisation de la démocratie en Europe Occidentale, faisant un bon nombre de pas vers des institutions plus influencées par le peuple.
Nous allons nous intéresser ici à la Première République Française associée à la période correspondant à l'ensemble des régimes républicains en France qui arrive grâce à la révolution française (1789-1799). \nocite{wiki:premiererep}

% TODO explication de comment les gens étaient elu / Y a des trucs a rajouter

\chapter{Méthodes d'élection}
\section{France} % TODO rename

\section{Suffrage indirect aux États-Unis}

Ce système d'élection est particularisé par la présence d'intermédiaires entre les électeurs du peuple et les concurrents. 
C'est à dire que les citoyens votent non pas directement pour les candidats mais plutôt pour des "grands électeurs" qui eux choisiront ultimement qui obtiendra le pouvoir.
Cette organisation, initialement préférée dans les jeunes démocraties, permettant de maintenir au pouvoir des personnes moins populaires mais aimées de ceux déjà en haut au détriment du peuple peu informé, perd progressivement en réputation avec l'éducation des citoyens et l'accessibilité croissante à l'information.
\nocite{wiki:scrutinindir}

La V\up{ème} République a employé cette méthode de suffrage pendant sa première élection, avant que le \textit{referendum} proposé par Charles De Gaulle ne soit validé et un suffrage direct établi. \nocite{polmania:scrutins}

\subsection{Éligibilité et Historique}

Tout d’abord, il y a des prérequis afin d’être président des États-Unis.
Il faut être âgé de minimum 35 ans, citoyen des États-Unis à sa naissance, avoir résidé aux États-Unis plus de 14 ans et ne pas être à son troisième mandat.
Ces dernières années, on décèle deux partis proéminents, bien qu'il y ait des candidats indépendants; le parti Démocrate et le parti Républicain.

Les élections présidentielles existent dans ce pays depuis 1789.
Cependant les lois furent un petit peu altérées en 1803 mais l'organisation actuelle est celle des lois de 2009.
Les changement de lois sont internes aux états donc le système de vote reste le même, la constitution demeure inchangée.
Un point historique important est aussi la fixation du “\textit{Election day}”, le premier mardi du mois de novembre.
C’est le jour fixé par la loi pour l'élection du suffrage universel.

Bien que certains votent le jour-même, nombreux sont ceux qui le font par correspondance.
Militaire en service, personne à mobilité réduite ou simplement pour alléger la charge le jour J, il y a une variété de raison pour lesquelles certains préfèrent voter en avance.
En 2020, tandis que la pandémie ravage le pays, ce sont plus de 100 millions de votes par correspondance qui sont comptabilisés, soit environ deux tiers des votes. \nocite{electproj:electproj}

\subsection{Fonctionnement} % TODO make paragraphes
Le modèle de vote pour les élections présidentielles aux états unis est basé sur un scrutin dit “indirect”.
Ce modèle de vote est un système d'élection ou les électeurs ne choisissent pas eux même la personne qui va être élue mais ils élisent des personnes qui vont faire ce choix.
Dans le cadre des États-Unis, chaque État possède un certain nombre d'électeurs.
Ces électeurs sont comptés au nombre de 538 répartis entre chaque État.
Prenons pour exemple l'État d'Ohio qui possède 18 grands-électeurs.
Si le vote des citoyens et majoritairement pour le parti démocrate.
Le parti démocrate empoche alors 18 votes car il y a la politique du “winner takes all” qui explique que la majorité des votes emporte l'entièreté des grands électeurs d’un état.
Ceci dit, il n’est pas non plus interdit qu’un grand électeur vote pour autre chose que la majorité choisie par les citoyens, mais il risque de grosses amendes pour candidat déloyal dans la majorité des états.
Pour gagner les élections présidentielles américaines, il faut que le candidat obtienne plus de 270 votes de la part des grands électeurs.
Si jamais la majorité absolue n’est pas atteinte par l’un des deux candidats, c’est alors la Chambre des représentants qui élit le président et le Sénat qui désigne le Vice-président.
\nocite{wiki:electday}
\nocite{wiki:elecus}
\nocite{wiki:eleccoll}

\section{Inde} % TODO

\section{Systèmes divers} % TODO


\chapter{Impact sur la démocratie}
% TODO intro

\section{France} % TODO rename
Nous avons que la France emploie un scrutin dans lequel les électeurs votent initialement pour un seul candidat parmi plusieurs.
Si la majorité absolue (au moins 50\% des voix) n’est pas atteinte par un des candidats, une second tour est mis en place avec les deux candidats ayant reçu le plus de voix du premier tour.
Cette façon de faire présente des avantages mais aussi des inconvénients.

\subsection{Condition de participation au vote}

Comme vu précédemment, le procédé de vote comprenait des conditions plutôt précises.

Voyons tout d'abord ce qu'il en est pour les électeurs.
En effet, les femmes notamment n’avaient jusqu'en 1944 \nocite{wiki:droitvotefemmes} pas le droit de vote.
Le système de vote actuel attend donc de l'électeur qu’il soit de nationalité française, âgé d'au moins 18 ans révolus et n’étant dans aucun cas d’incapacité prévu par la loi.

Du côté des candidats, ces derniers doivent être âgés de 18 ans, ne pas être privés de leurs droits d’éligibilité par une décision de justice, ne pas être placés sous tutelle ou curatelle, avoir satisfait aux obligations imposées par le code du service national et avoir fait preuve de “dignité morale” mais cette partie n’est pas définie précisément.
Pour finir les candidats devront disposer de 500 signatures provenant de différents départements avec des proportions minimales à respecter.

\subsection{Déroulement du Scrutin}

La France, pour la démocratie, fait le choix d'utiliser un scrutin en plusieurs tours, s'il s'en avère nécessaire.
Tout d’abord une multitude de candidats se présentent.
Parmi ceux-ci, qui ont tous rempli les conditions vu ci-dessus, le peuple vote pour l’un d’entre eux dans leurs bureaux de vote respectifs.
Les électeurs votent uniquement pour un candidat.
Une fois le dépouillement terminé et les totaux de vote comptabilisés, si aucun candidat n’obtient la majorité des votes, c’est un second tour qui se lance opposant les deux candidats qui ont obtenu le plus de votes au premier tour.

\section{Élections présidentielles aux États-Unis}
% TODO re-intro

\subsection{Avantages / Désavantages} % TODO rename & revamp
Ce système de vote montre des avantages et des inconvénients.
Un avantage est la Représentation régionale plus équitable.
On donne ici une force plus ou moins équitable à chaque État, petit ou grand.
Le problème de cet avantage est aussi un inconvénient.
Si un petit État possède une force plus ou moins égale à celle dans un plus grand État, chaque habitant ne possède pas le même impact sur l'élection du président.
La voix d’un habitant d’un tout petit état vaudra alors beaucoup plus que quelqu’un dans un grand état. *TODO ex cal*
Cela apporte donc un nouveau désavantage qui est qu’un candidat peut passer à la tête d’un pays alors qu’il ne possède pas la majorité du vote populaire, ce qui s’est produit en 1824, 1876, 1888, 2000 et 2016.
Un autre avantage de ce système est le résultat.
Si jamais il y a des problèmes de fraudes en termes de votes ou quoi que ce soit en terme de doute, il n’est pas nécessaire de recompter les votes de l'entièreté du pays mais uniquement les états dans lesquels le doute est posé.
Un dernier problème est l’importance posée sur les swings states.
En effet, une majorité d’états a de très faibles chance de changer leur opinion politique et donc le vote final.
Ces états sont donc source d’un autre problème qui est que les habitants de ces États n’ont pas le sentiment d’avoir un vote utile étant donné que le résultat a de forte chance d’être le même d’années en années.
Cependant, d’autres états eux ont une opinion politique majoritaire qui a tendance à changer et sont appelés les swing states.
Ces swing states étant donné qu’ils sont minoritaires ont un impact majeur sur les élections présidentielles ce qui représente un problème.
Ces swing states qui ne représentent pas l'entièreté de la nation détiennent donc trop de pouvoir électoral.
\nocite{greelane:eleccoll}
\nocite{gov:fedpapers68}

\begin{figure}[h]
	\centering
	\includegraphics[width=0.85\textwidth]{./images/2012-election.png}
	\caption{Carte prévisionnel de l'élection 2012, kqed.com \cite{kqed:map}}
	\label{map:uselec2012}
\end{figure}

Cette carte (fig. \ref{map:uselec2012}) représente en rouge les prévisions des élections présidentielles américaines de 2012 avec en beige les fameux "swing states".
On voit donc bien qu'une petite partie des états-unis détient la totalité du pouvoir électoral étant donné que les votes entre républicains (rouge) et les démocrate (bleu) sont plutôt égaux et qu'il manque donc le choix des swing states pour déterminer qui passera au pouvoir.

\begin{figure}[h]
	\centering
	\includegraphics[width=0.4\textwidth]{./images/spoiler-effect.png}
	\caption{Exemple de l'effet spoiler, via ncase.me \cite{ncase:ballot}}
	\label{sim:spoilereffect}
\end{figure}


On a ici un exemple d'effet "spoiler" (fig. \ref{sim:spoilereffect}).
Cet effet montre que s'il y a deux candidats du même parti comme par exemple le triangle et l'hexagone.
Ces deux derniers possèdent beaucoup plus de votes à eux deux que le carré mais pourtant le carré est gagnant de cette élection étant donné que la majorité est pour le triangle et l'hexagone se divise en deux pour chaque candidat, ce qui donne un résultat inférieur à celui du carré.
Ici le carré ne possède pas du tout la majorité des votes (seulement ~35\%) mais va gagner les élections.
C'est arrivé en 2000 quand Ralph Nader et Al Gore se sont partagé les votes ce qui a permis à George Bush de gagner.


\section{Inde} % TODO ??

\section{Coût des élections}

\subsection{Défi logistique}
\subsection{Investissement de l'État}
\subsection{L'Économie de la démocratie}

Les économies globales comme locales sont impactées par les périodes électorales.

Par exemple, sur le marché de l’immobilier durant les périodes électorales, les investisseurs attendent plutôt de connaître quelles directives le nouveau gouvernement suivra et seulement une fois l'élection passée investirons, ce qui crée des baisses d’activité économique dans ce secteur.

Dans le reportage offert par France 3\nocite{fr3:impact}, un coiffeur qui tient son salon depuis 12 ans affirme que c’est très fréquent que survienne une baisse de son chiffre d’affaire (entre 10 et 15\%) dans les semaines précédant l'élection.

Les scrutins pourraient réduire ces effets nuisant à l'économie notamment en diminuant la durée des élections. Un scrutin par classement éliminerait le deuxième tour et réduirait ce délai.

% TODO investissements entreprises etc
% Peut-être intégrer dans les autres sections


\newpage
\section*{Conclusion}
TODO

\bibliography{bibliography} 
\bibliographystyle{ieeetr}

\end{document}
